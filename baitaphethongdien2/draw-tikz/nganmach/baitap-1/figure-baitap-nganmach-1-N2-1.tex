\documentclass[border=12pt]{standalone}
\usepackage[utf8]{inputenc}
\usepackage[utf8]{vietnam} %Bien dich duoc tieng Viet
\usepackage{amsmath,amsfonts,amssymb} %Font toan
\usepackage[american,cuteinductors,smartlabels]{circuitikz}
\begin{document}
\begin{circuitikz}\draw
	%Nhánh  máy phát 1
	(0,0) to [L,*-,l_ = $X_9$] (0,2) to [short] (0,4) to [L,-*, l_ = $X_3$] (0,6)
								
	%Nhánh máy phát 2
	(3,0) to [L,*-, l_ = $X_9$] (3,2) to [L, l_ = $X_7$] (3,4) to [L,*-*, l_ = $X_5$] (3,6)

	%Nhánh máy phát 3
	(6,0) to [L,*-, l_ = $X_9$] (6,2) to [L, l_ = $X_7$] (6,4) to [L,*-*, l_ = $X_5$] (6,6)

	%Nhánh máy phát 4
	(10,0) to [L,*-,l_ = $X_9$] (10,2) to [L,-*, l_ = $X_4$] (10,6)

	% Cuộn dây của máy biến áp tự ngẫu
	(3,4) to [short,*-*] (6,4) to [short,*-] (9,4) to [short,-*] (9,6)

	%Liên kết các nhánh của các máy biến biến áp 
	(-1,6) to [short] (7,6)
	(8,6) to [short] (11,6)

	%Đường dây kép
	(1.5,6) to [L,*-*, l_ = $X_2$] (1.5,8)
	(4.5,6) to [L,*-*, l_ = $X_2$] (4.5,8)

	%Nối 2 đường dây kép với hệ thống
	(.5,8) to [short] (5.5,8)

	%Vẽ hệ thống
	(3,8) to [L,*-*, l_ = $X_1$] (3,10)

	%Vẽ điểm ngắn mạch
	(8.5,6.8) to [open,l_ = $\mathbf{N_2}$] (9,6.8)
	;
\end{circuitikz}
\end{document} 
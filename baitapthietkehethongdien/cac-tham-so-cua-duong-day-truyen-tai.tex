\section{Tính toán các tham số của đường dây truyền tải}
\subsection{Đường dây truyền tải}
	Với đường dây truyền tải, ta cần khảo sát các thông số: \emph{điện áp}, \emph{dòng điện}, \emph{công suất} và \emph{hệ số công suất} ở \emph{đầu gửi} và \emph{đầu nhận}.\\

	Việc\emph{ lựa chọn đường dây truyền} tải theo yêu cầu: \emph{tổn thất công suất nhỏ nhất}, \emph{hiệu quả cao trong vận hành} và \emph{độ sụt áp trong giới hạn cho phép}.\\
	
	\emph{Phân loại đường dây truyền tải:}
		\begin{itemize}
			\item \emph{Đường dây ngắn:} chiều dài $l <80\unit{km}$.
			
			\item \emph{Đường dây trung bình:} chiều dài $80 \unit{km} \leq l \leq 240 \unit{km}$.
			
			\item \emph{Đường dây dài:} chiều dài $l > 240 \unit{km}$.
		\end{itemize}
		
	\emph{Thông số đường dây:}
		\begin{itemize}
			\item \emph{Tổng trở:} $\overline{Z} = \pfm{r_0+jx_0}l = \pfm{r_0+j\omega L_0}l$.
			\item \emph{Tổng dẫn:} $\overline{Y} = \pfm{g_0+jb_0}l \approx jb_0l =  j\omega C_0 l$.
		\end{itemize}
		
	Gọi $V_S$ và $V_R$ lần lượt là \emph{điện áp pha} đầu gửi và \emph{điện áp pha} đầu nhận.
	\begin{itemize}
		\item Phần trăm sụt áp: $\Delta U \%= \dfrac{V_S - V_R}{V_R} \times 100 \%$

		\item Các công thức tính công suất:
			\begin{align*}
				P & = \sqrt{3}U_{L}I_{L}\cos \varphi = 3U_{P}I_{P}\cos \varphi; &  Q & = \sqrt{3}U_{L}I_{L}\sin \varphi = 3U_{P}I_{P}\sin \varphi;\\
			S & = \sqrt{3}U_{L}I_{L} = 3 U_{P} I_{P}; & \overline{S} & = \sqrt{3}\overline{U}_{L}\overline{I}_{L}^{\ast} = 3\overline{U}_{P}\overline{I}_{P}^{\ast} = P + jQ\\
			P  & = S\cos \varphi; \quad Q  = S\sin \varphi =  P\tan\varphi;  & S & = \sqrt{P^2 + Q^2}
			\end{align*}					
		
		\item Tổn hao công suất trên đường dây truyền tải: $\Delta P = P_S - P_R$.
		
		\item Hiệu suất của đường dây: $\eta = \dfrac{P_R}{P_S} \times 100\%$
		
		\item Hệ số công suất: $\cos \varphi = \cos \pfm{\varphi_{V} - \varphi_{I}}$.
	\end{itemize}
	
	\emph{Mô hình hóa của đường dây truyền tải:}
		\begin{itemize}		
			\item Mô hình hóa: hình \ref{Fig:mo-hinh-hoa-duong-day}.
			\begin{figure}[!h]
				\begin{center}
					\begin{tikzpicture}
					\draw[-triangle 90] (0,0) -- (1,0);
					\draw (1,0) -- (2,0);
					\draw (2,.4) rectangle (5,-1);
					\draw[-triangle 90] (5,0) -- (6,0);
					\draw (6,0) -- (7,0);
				
					\draw (0,-.6) -- (2,-.6);
					\draw (5,-.6) -- (7,-.6);
					\draw (0.5,0.3) node{$I_S$} ;
					\draw (0.5,-0.4) node{$V_S$} ;
				
					\draw (6.5,0.3) node{$I_R$} ;
					\draw (6.5,-0.4) node{$V_R$} ;
				
					\draw (3.5,-0.3) node{$\overline{A}, \overline{B}, \overline{C}, \overline{D}$};	
				\end{tikzpicture}
			\end{center}
			\caption{Mô hình hóa của đường dây truyền tải} \label{Fig:mo-hinh-hoa-duong-day}
		\end{figure}
		
		\item Phương trình quan hệ dòng điện và điện áp giữa đầu gửi với đầu nhận:
			\begin{align*}
				\left[{\begin{array}{c}
				\overline{V}_S\\
				\overline{I}_S
				\end{array}
				}\right]
				= 
				\left[{\begin{array}{cc}
				\overline{A} & \overline{B}\\
				\overline{C} & \overline{D}
				\end{array}
				}\right]				
				\left[{\begin{array}{c}
				\overline{V}_R\\
				\overline{I}_R
				\end{array}
				}\right]
				\Longleftrightarrow
				 \left\{{\begin{array}{c}
				 \overline{V}_S = \overline{A}.\overline{V}_R + \overline{B}.\overline{I}_R\\
				 \overline{I}_S = \overline{C}.\overline{V}_R + \overline{D}.\overline{I}_R
				 \end{array}
				}\right.
			\end{align*}
			
		\item Với các thông số $\overline{A}, \overline{B}, \overline{C}, \overline{D}$ được xác  định tùy theo \emph{loại đường dây} (đường dây ngắn, đường dây trung bình, đường dây dài) và  \emph{mô hình hóa} của đường dây (mạch $T$ chuẩn, mạch $\Pi$ chuẩn).
	\end{itemize}

\subsection{Công thức tổng quát của mô hình đường dây truyền tải}
	\begin{itemize}
		
		\item Sử dụng công thức tổng quát cho kết quả chính xác hơn so với các công thức gần đúng (các công thức của đường dây trung bình).
		
		\item Công thức tổng quát khai triển \emph{3 số hạng đầu} trong khai triển \emph{Maclaurin}:
			\begin{align*}						
				\overline{A}  = \overline{D} = 1 + \dfrac{\overline{Y}.\overline{Z}}{2} + \dfrac{\overline{Y}^2.\overline{Z}^2}{24}; \quad \overline{B}  = \overline{Z}\pfm{1 + \dfrac{\overline{Y}.\overline{Z}}{6} + \dfrac{\overline{Y}^2.\overline{Z}^2}{120}}; \quad
				 \overline{C} = \overline{Y} \pfm{1 + \dfrac{\overline{Y}.\overline{Z}}{6} + \dfrac{\overline{Y}^2.\overline{Z}^2}{120}}	
			\end{align*}
		\end{itemize}
\subsection{Đường dây truyền tải ngắn}
	\begin{itemize}
		\item Chiều dài đường dây: $l < 80 \unit{km}$.
		
		\item Giá trị của thông số $\overline{A}, \overline{B}, \overline{C}, \overline{D}$ trong mạch tương đương \emph{tổng trở nối tiếp}:
		\begin{align*}
			\overline{A} = \overline{D} = 1; \qquad \overline{B} = \overline{Z}; \qquad \overline{C} = 0
		\end{align*}
		
		\item Sơ đồ tương đương cho đường dây ngắn: hình \ref{Fig:mach-tuong-duong-duong-day-ngan}.
			\begin{figure}[!h]
			\begin{center}				
				\begin{circuitikz}
					\draw (0,0) to [european resistor, *-, l_ = $R$] (3,0) to [L, l_ = $jX$] (6,0) to [short, i_ = $ $, l_ = \text{$\dot{I} = \dot{I}_{S} = \dot{I}_{R}$}] (9,0) to [european resistor, l_ = $Load$] (9,-3) to [short, -*] (0,-3);
					\draw (.5,0) to [open, l_= \text{$\dot{V}_S$}] (.5,-3);
					\draw (6,0) to [open, l_= \text{$\dot{V}_R$}] (6,-3);
					\draw[<->] (0.5,-.2) -- (.5,-2.8);% node[below] {$\dot{V}_S$};	
					\draw[<->] (6,-.2) -- (6,-2.8);%node[below] {$\dot{V}_R$};					
				\end{circuitikz}
			\end{center}
			\caption{Mạch tương đương cho đường dây ngắn (pha với trung tính)} \label{Fig:mach-tuong-duong-duong-day-ngan}
			\end{figure}
	\end{itemize}
	
\subsection{Đường dây truyền tải trung bình}
	\begin{itemize}
		\item Chiều dài đường dây: $80 \unit{km} \leq l \leq  240 \unit{km}$.
		\item Giá trị của thông số $\overline{A}, \overline{B}, \overline{C}, \overline{D}$
			\begin{itemize}	
				\item Mạch tương đương $T$ \emph{chuẩn}:
					\begin{align*}
						\overline{A} = \overline{D} = 1 + \dfrac{\overline{Y}. \overline{Z}}{2}; \qquad \overline{B} = \overline{Z}\pfm{1 + \dfrac{\overline{Y} . \overline{Z}}{4}}; \qquad \overline{C} = \overline{Y}
					\end{align*}
		
			\item Mạch tương đương $\Pi$ \emph{chuẩn}:
				\begin{align*}
					\overline{A} = \overline{D} = 1 + \dfrac{\overline{Y}. \overline{Z}}{2}; \qquad \overline{B} = \overline{Z}; \qquad \overline{C} = \overline{Y}\pfm{1 + \dfrac{\overline{Y} . \overline{Z}}{4}}
				\end{align*}
			\item Sơ đồ tương đương cho đường dây trung bình:
				\begin{itemize}			
				\item Sơ đồ tương đường hình $T$ chuẩn: hình \ref{Fig:mach-tuong-duong-duong-day-trung-binh-T}.
			\begin{figure}[!h]
			\begin{center}				
				\begin{circuitikz}
					\draw (-1,0) to [short, *-] (0,0) to [european resistor, l_ = $\frac{1}{2}R$] (2,0) to [L, l_ = $j\frac{1}{2}X$] (4,0) to [short] (6,0) to [european resistor, l_ = $\frac{1}{2}R$] (8,0) to [L, l_ = $j\frac{1}{2}X$] (10, 0) to [short] (12, 0) to [european resistor, l_ = $Load$] (12,-4) to [short, -*] (-1,-4);
					\draw (5,0) to [C, l_=\text{$\dot{Y} = jB$}] (5,-4);
					\draw (5,0) to [short, i_ = $\dot{I}_R$] (6.4,0);
					\draw (-1,0) to [short, i_ = $\dot{I}_S$] (0.3,0);
					\draw (5,-2.5) to [short, i_ = $\dot{I}_C$] (5,-4);
					\draw (0,0) to [open, l_= \text{$\dot{V}_S$}] (0,-4);
					\draw (10,0) to [open, l_= \text{$\dot{V}_R$}] (10,-4);
					\draw[<->] (0,-.2) -- (0,-3.8);% node[below] {$\dot{V}_S$};	
					\draw[<->] (10,-.2) -- (10,-3.8);% node[below] {$\dot{V}_R$};					
				\end{circuitikz}
			\end{center}
			\caption{Mạch tương đương hình $T$ chuẩn cho đường dây trung bình} \label{Fig:mach-tuong-duong-duong-day-trung-binh-T}
			\end{figure}
			
			\item Sơ đồ tương đường hình $\Pi$ chuẩn: hình \ref{Fig:mach-tuong-duong-duong-day-trung-binh-Pi}.
			\begin{figure}[!h]
			\begin{center}				
				\begin{circuitikz}
					\draw(-3,0) to [short,*-] (0,0) to [european resistor, l_ = $R$] (3,0) to [L, l_ = $jX$] (6,0) to [short] (8,0) to [short, i_ = $ $, l_ = \text{$\dot{I}_{R}$}] (10,0) to [european resistor, l_ = $Load$] (10,-4) to [short, -*] (-3,-4);
					\draw (6,0) to [C, l_=\text{$\frac{\dot{Y}}{2} = j\frac{B}{2}$}] (6,-4);
					\draw (6,-2.5) to [short, i_ = $\dot{I}_C$] (6,-4);
					\draw (0,0) to [C, l_=\text{$\frac{\dot{Y}}{2} = j\frac{B}{2}$}] (0,-4);
					\draw (-2.5,0) to [open, l_= \text{$\dot{V}_S$}] (-2.5,-4);
					\draw (8,0) to [open, l_= \text{$\dot{V}_R$}] (8,-4);
					\draw[<->] (-2.5,-.2) -- (-2.5,-3.8);% node[below] {$\dot{V}_S$};	
					\draw[<->] (8,-.2) -- (8,-3.8);%node[below] {$\dot{V}_R$};					
				\end{circuitikz}
			\end{center}
			\caption{Mạch tương đương hình $\Pi$ chuẩn cho đường dây trung bình} \label{Fig:mach-tuong-duong-duong-day-trung-binh-Pi}
			\end{figure}
			\end{itemize}
		\end{itemize}
	\end{itemize}
	
\subsection{Đường dây truyền tải dài}
	\begin{itemize}
		\item Chiều dài đường dây: $l >  240 \unit{km}$.
		
		\item Giá trị của thông số $\overline{A}, \overline{B}, \overline{C}, \overline{D}$ trong mạch tương đương  \emph{thông số rãi}:
		\begin{align*}
			\overline{A} = \overline{D} = \cosh \theta; \qquad \overline{B} = \overline{Z}_C \sinh \theta; \qquad \overline{C} = \dfrac{\sinh\theta}{\overline{Z}_C}
		\end{align*}
		với $\theta = l\sqrt{\overline{z}.\overline{y}}$ và $\overline{Z}_C = \sqrt{\dfrac{\overline{z}}{\overline{y}}}$.
		\item Một số \emph{công thức liên quan đến số phức} cần cho mô hình tính toán đường dây dài:
			\begin{itemize}		
		 		\item[$\ast$] Căn bậc hai của số phức: $w = \pfm{r\angle \varphi}^\frac{1}{2} = \sqrt{r} \angle \dfrac{\varphi}{2}$
		 		\item[$\ast$] Hàm lượng giác \emph{hyperbolic} với số phức (khi tính toán chuyển sang chế độ \emph{radian}): 
		 			\begin{align*}
		 				\sinh \pfm{a+jb} & = \sinh a \cos b + j\cosh a \sin b\\
		 				\cosh \pfm{a+jb} & = \cosh a \cos b + j\sinh a \sin b
		 			\end{align*}
				\item[$\ast$] Hàm lượng giác của hàm \emph{hyperbolic}:
			\begin{align*}
				\sinh \theta = \dfrac{e^\theta- e^{-\theta} }{2}; \qquad \cosh  \theta = \dfrac{e^\theta + e^{-\theta} }{2}
			\end{align*}
		\end{itemize}
		\item Sơ đồ tương đương cho đường dây dài: hình \ref{Fig:mach-tuong-duong-duong-day-dai}.
			\begin{figure}[!h]
			\begin{center}				
				\begin{circuitikz}
					\draw (0,0) to [short, *-, i_ = $I_S$] (1, 0) to [european resistor, l_ = $r$] (3,0) to [L, l_ = $jx$] (5,0) to [short] (6.5,0) to [european resistor, l_ = $r$] (8.5,0) to [L, l_ = $jx$] (10.5,0) to [short] (12,0) to [european resistor, l_ = $r$] (14,0) to [L, l_ = $jx$] (16,0) to [short, i_ = $\dot{I}_S$] (18,0) to [european resistor, l_=$Load$] (18,-4) to [short, -*] (0,-4);
					\draw (5.75, 0) to [short] (5.75,-1) to [short] (6.75,-1) to [C, l_ = $jb$] (6.75,-3) to [short] (4.75,-3) to [european resistor, l_=$g$] (4.75,-1) to [short] (5.75,-1);
					\draw (5.75,-3) to [short] (5.75,-4);
					
					\draw (11.25, 0) to [short] (11.25,-1) to [short] (12.25,-1) to [C, l_ = $jb$] (12.25,-3) to [short] (10.25,-3) to [european resistor, l_=$g$] (10.25,-1) to [short] (11.25,-1);
					\draw (11.25,-3) to [short] (11.25,-4);
					
					\draw (1,0) to [open, l_= \text{$\dot{V}_S$}] (1,-4);
					\draw (16,0) to [open, l_= \text{$\dot{V}_R$}] (16,-4);
					\draw[<->] (1,-.2) -- (1,-3.8);% node[below] {$\dot{V}_S$};	
					\draw[<->] (16,-.2) -- (16,-3.8);%node[below] {$\dot{V}_R$};					
				\end{circuitikz}
			\end{center}
			\caption{Các thông số rãi của đường dây truyền tải dài} \label{Fig:mach-tuong-duong-duong-day-dai}
			\end{figure}
	\end{itemize}
\subsection{Bài tập tính toán các tham số của đường dây truyền tải}
	\begin{enumerate}
		\item \label{Ex-tham-so-duong-day:bt1} Một đường dây 3 pha, $11 \unit{kV}$, dài $10 \unit{km}$ chuyển cho đầu nhận tải $5000 \unit{kW}$, $\cos \varphi_R = 0.8$ (trễ). Điện trở mỗi pha trên một $km$ là $0.1 \unit{\Omega}$ và cảm kháng mỗi pha trên một $km$ là $0.2 \unit{\Omega}$. Tính:
			\begin{enumerate}[a.]
				\item Vẽ sơ đồ thay thế cho mô hình đường dây ngắn.
				\item Các thông số $\overline{A}, \overline{B}, \overline{C}, \overline{D}$ của đường dây.
				\item Điện áp và dòng điện đầu gửi.
				\item Hệ số công suất đầu gửi.
				\item Góc lệnh pha giữa điện áp đầu gửi và đầu nhận.
				\item Độ sụt áp.
				\item Công suất tác dụng, phản kháng và biểu kiến ở đầu gửi.
				\item Tổn thất công suất trên đường dây.
				\item Hiệu suất truyền tải.
				\item Vẽ giản đồ vector.
				\item[$\ast$] \emph{Lưu ý:} Làm tròn kết quả \emph{2 chữ số} sau dấu phẩy.
			\end{enumerate}
			
		\subparagraph{Bài giải bài tập \ref{Ex-tham-so-duong-day:bt1}}
			\begin{enumerate}[\it a.]
				\item[$\bullet$] Ta có: $\overline{Z} = \pfm{r_0 + j x_0}l = \pfm{0.1 +j 0.2}\times 10 = 1 + j 2 = 2.24 \angle 63.43^0 \unitp{\Omega}$.
					
				\item[$\bullet$] Chọn $\overline{V}_R = \dfrac{11}{\sqrt{3}} \angle 0^0 = 6.35 \angle 0^0 \unitp{kV} $
				
				\item[$\bullet$] Suy ra: $I_R = \dfrac{P_R}{\sqrt{3}V_R \cos \varphi} = \dfrac{5\times 10^3}{\sqrt{3} \times 11 \times 0.8} = 0.33 \times 10^3 \unit{A} = 0.33\unitp{kA}$.
				
				\item[$\bullet$] Có $\cos \varphi_R = 0.8$ (trễ) $ \Longrightarrow \varphi_R  = +36.87^0$.				
				\item[$\bullet$] Có $\varphi_R = \varphi_{V_R} - \varphi_{I_R} \Longrightarrow \varphi_{I_R} = \varphi_{V_R} - \varphi_R = 0^0 - 36.87^0 = -36.87^0$.
				\item[$\bullet$] Nên: $\overline{I}_R = 0.33 \angle -36.87^0 \unitp{kA}$ .

				\item \emph{Sơ đồ tương đương cho đường dây ngắn:} hình \ref{Fig:mach-tuong-duong-duong-day-ngan-bt1}.
			\begin{figure}[!h]
			\begin{center}				
				\begin{circuitikz}
					\draw (0,0) to [european resistor, *-, l_ = $1\unit{\Omega}$] (3,0) to [L, l_ = $j2 \unit{\Omega}$] (6,0) to [short, i_ = $ $, l_ = \text{$\dot{I} = \dot{I}_{S} = \dot{I}_{R}$}] (9,0) to [european resistor, l_ = $Load$] (9,-3) to [short, -*] (0,-3);
					\draw (.5,0) to [open, l_= \text{$\dot{V}_S$}] (.5,-3);
					\draw (6,0) to [open, l_= \text{$\dot{V}_R$}] (6,-3);
					\draw[<->] (0.5,-.2) -- (.5,-2.8);% node[below] {$\dot{V}_S$};	
					\draw[<->] (6,-.2) -- (6,-2.8);%node[below] {$\dot{V}_R$};					
				\end{circuitikz}
			\end{center}
			\caption{Mạch tương đương cho đường dây ngắn trong bài tập \ref{Ex-tham-so-duong-day:bt1}} \label{Fig:mach-tuong-duong-duong-day-ngan-bt1}
			\end{figure}				
				\newpage
				\item \emph{Thông số $\overline{A}, \overline{B}, \overline{C}, \overline{D}$ cho mô hình đường dây ngắn}				
					\begin{align*}
						\overline{A} 	= \overline{D}  = 1; \qquad \overline{B} = \overline{Z} = 2.24 \angle 63.43^0 \unitp{\Omega}; \qquad \overline{C} = 0
					\end{align*}
				
				\item \emph{Xác định điện áp đầu gửi $\overline{V}_S$ và dòng điện đầu gửi $\overline{I}_S$}
					\begin{itemize}
						\item Ta có: 
							\begin{align*}
								\left[{\begin{array}{c}
								\overline{V}_S\\
								\overline{I}_S
								\end{array}
								}\right]
								= 
								\left[{\begin{array}{cc}
								\overline{A} & \overline{B}\\
								\overline{C} & \overline{D}
								\end{array}
								}\right]				
							\left[{\begin{array}{c}
							\overline{V}_R\\
							\overline{I}_R
							\end{array}
							}\right]
							= \left[{\begin{array}{cc}
								1 & 2.24 \angle 63.43^0\\
								0 & 1
								\end{array}
								}\right]				
							\left[{\begin{array}{c}
							6.35 \angle 0^0\\
							0.33 \angle -36.87^0
							\end{array}
							}\right]
						\end{align*}
						
						\item Điện áp đầu gửi:
							\begin{align*}
								\overline{V}_S  = \overline{A}. \overline{V}_R + \overline{B}.\overline{I}_R = 1 \times 6.35 \angle 0^0 + 2.24 \angle 63.43^0 \times 0.33 \angle -36.87^0 = 7.02 \angle 2.70^0 \unitp{kV}
								\end{align*}
								
						\item Điện áp dây đầu gửi: $V_{LS}  = \sqrt{3} V_R = \sqrt{3} \times 7.02 = 12.16 \unitp{kV}$.
						\item Dòng điện đầu gửi:
							\begin{align*}								
								\overline{I}_S  = \overline{C}. \overline{V}_R + \overline{D}.\overline{I}_R = 0 \times 6.35 \angle 0^0 + 1 \times 0.33 \angle -36.87^0 = 0.33 \angle -36.87^0 \unitp{kA}
							\end{align*}
					\end{itemize}

					\item \emph{Xác định hệ số công suất đầu gửi $\cos \varphi_S$}
						\begin{align*}
							\cos \varphi_S  = \cos \pfm{\varphi_{V_s} - \varphi_{I_{s}}}= \cos \left[{2.70^0 - \pfm{-36.87^0}} \right] = \cos 39.57^0 = 0.77
						\end{align*}
				
				\item \emph{Xác định góc lệch pha giữa điện áp đầu gửi và đầu nhận $\Delta \varphi_V$}
					\begin{align*}
						\Delta \varphi_V = \varphi_{V_S} - \varphi_{V_R} = 2.70^0 - 0^0 = 2.70^0
					\end{align*}
					
				\item \emph{Xác định độ sụt áp $\Delta U$}
					\begin{align*}
						\Delta U \%= \dfrac{V_S - V_R}{V_R} \times 100 \% = \dfrac{7.02 - 6.35}{6.35} \times 100 \% = 10.55\%
					\end{align*}
					
				\item \emph{Xác định công suất tác dụng, phản kháng và biểu kiến ở đầu gửi $P_S, Q_S, S_S$}
					\begin{align*}
						& \overline{S}_S  = 3 \overline{V}_S.\overline{I}_S^{\ast} = 3 \times 7.02 \angle 2.70^0 \times 0.33 \angle +36.87^0 = 5.36 + j4.43 \unitp{MVA}\\
						\Longrightarrow & P_S  = 5.36 \unit{MW}; \qquad Q_S = 4.43 \unit{MVAr}
					\end{align*}
				\item \emph{Xác định tổn thất công suất $\Delta P$}
					\begin{align*}
						\Delta P = P_S - P_R = 5.36 - 5 = 0.36 \unitp{MW}
					\end{align*}
				
				\item \emph{Xác định hiệu suất $\eta$}
						\begin{align*}
							\eta = \dfrac{P_R}{P_S} \times 100 \% = \dfrac{5}{5.36} \times 100\% = 93.28\%
						\end{align*}
					
				\item \emph{Vẽ giản đồ vector:} hình \ref{Fig:gian-do-vector-bt1}.
					\begin{figure}[!h]
						\begin{center}
							\begin{tikzpicture}
								\coordinate (origo) at (0,0);
								\coordinate (pivot) at (1,5);
    
    							\draw (0,0) node[left] {$0$};
    							\draw[thick,->] (origo) -- ++(4,0) node (VR) [right] {$V_R$};

								\draw[thick,->] (origo) -- ++(330:3) coordinate (IR) node[right] {$I_R =  I_S$};
    
							    \draw[thick,->] (origo) -- ++(370:4.5) coordinate (VS) node[right] {$V_S$};

								\pic [draw, <->, "$\varphi_R$", angle eccentricity=1.7] {angle = IR--origo--VR};
    
								\pic [draw, <->, "$\varphi_S$", angle eccentricity=1.2, angle radius=1.5cm] {angle = IR--origo--VS};
    
							    \pic [draw, <->, "$\Delta \varphi_V$", angle eccentricity=1.2, angle radius=2.5cm] {angle = VR--origo--VS};
							\end{tikzpicture} \hspace{1cm}							
							 \begin{tikzpicture}
    							\coordinate (origo) at (0,0);
   								\coordinate (pivot) at (1,5);
    
    							\draw (0,0) node[left] {$0$};
							    \draw[thick,->] (origo) -- ++(4,0) node (IR) [right] {$I_R = I_S$};

							    \draw[thick,->] (origo) -- ++(387:3) coordinate (VR) node[right] {$V_R$};
    
							    \draw[thick,->] (origo) -- ++(400:4.5) coordinate (VS) node[right] {$V_S$};

							    \pic [draw, <->, "$\varphi_R$", angle eccentricity=1.5, angle radius=.7cm] {angle = IR--origo--VR};
    
							    \pic [draw, <->, "$\varphi_S$", angle eccentricity=1.2, angle radius=1.7cm] {angle = IR--origo--VS};
    
								\pic [draw, <->, "$\Delta \varphi_V$", angle eccentricity=1.2, angle radius=2.5cm] {angle = VR--origo--VS};
						  \end{tikzpicture}
						\end{center}
						\caption{Giản đồ vector cho bài tập \ref{Ex-tham-so-duong-day:bt1}} \label{Fig:gian-do-vector-bt1}
					\end{figure}
			\end{enumerate}
		
		\item \label{Ex-tham-so-duong-day:bt2} Một đường dây 3 pha, $110 \unit{kV}$, $f=50Hz$, dài $150 \unit{km}$ chuyển cho đầu nhận tải $40000 \unit{kW}$, $\cos \varphi_R = 0.8$ (trễ). Điện trở mỗi pha trên một $km$ là $0.15 \unit{\Omega}$,  dung dẫn mỗi pha trên một $km$ là $10 \times 10^{-6} \unit{\Omega^{-1}}$ và cảm kháng mỗi pha trên một $km$ là $0.6 \unit{\Omega}$. Thực hiện tính toán trên 3 mô hình: \emph{mô hình $T$ chuẩn},  \emph{mô hình $\Pi$ chuẩn} và \emph{mô hình tổng quát}.
			\begin{enumerate}[a.]
				\item Vẽ sơ đồ thay thế cho mô hình đường dây trung bình cho mạch tương ứng.
				\item Các thông số $\overline{A}, \overline{B}, \overline{C}, \overline{D}$ của đường dây.
				\item Điện áp và dòng điện đầu gửi.
				\item Độ lệch pha giữa điện áp đầu gửi và đầu nhận.
				\item Hệ số công suất đầu gửi.
				\item Công suất tác dụng, phản kháng và biểu kiến ở đầu gửi.
				\item Tổn thất công suất trên đường dây.
				\item Độ sụt áp của đường dây.
				\item Hiệu suất truyền tải.
				\item Vẽ giản đồ vector.
				\item[$\ast$] \emph{Lưu ý:} Làm tròn kết quả \emph{2 chữ số} sau dấu phẩy.
			\end{enumerate}
			
		\subparagraph{Bài giải bài tập \ref{Ex-tham-so-duong-day:bt2}}
			\begin{enumerate}[ \it a.]
				\item[$\bullet$] Ta có:
					\begin{align*}
						\overline{Z} & = \pfm{r_0 + j x_0}l = \pfm{0.15 +j 0.6}\times 150 = 22.5 + j 90 = 92.77 \angle 75.96^0 \unitp{\Omega}\\
						\overline{Y} & = \pfm{g_0 + j b_0}l \approx j b_0 l = j 10 \times 10^{-6}\times 150 = j 1.5 \times 10^{-3} = 1.5 \times 10^{-3} \angle 90^0 \unitp{\Omega^{-1}}
					\end{align*}
				
				\item[$\bullet$] Chọn $\overline{V}_R = \dfrac{110}{\sqrt{3}} \angle 0^0 = 63.51 \angle 0^0 \unitp{kV} $
				\item[$\bullet$] Suy ra: $I_R = \dfrac{P_R}{\sqrt{3}V_R \cos \varphi_R} = \dfrac{40 \times 10^3} {\sqrt{3} \times 110 \times 0.8} = 0.26 \times 10^3 \unit{A} = 0.26\unitp{kA}$.
				
				\item[$\bullet$] Có $\cos \varphi_R = 0.8$ (trễ) $\Longrightarrow \varphi_R  = 36.87^0$.
				\item[$\bullet$] Có $\varphi_R = \varphi_{V_R} - \varphi_{I_R} \Longrightarrow \varphi_{I_R} = \varphi_{V_R} - \varphi_R = 0^0 - 36.87^0 = -36.87^0$.
				\item[$\bullet$] Nên: $\overline{I}_R = 0.26 \angle -36.87^0 \unitp{kA}$ .				
				
				\item[$\star$] \textbf{Mô hình $\mathbf{T}$ chuẩn}
				\item \emph{Sơ đồ tương đường hình $T$ chuẩn:} hình \ref{Fig:mach-tuong-duong-duong-day-trung-binh-T-bt2}.
			\begin{figure}[!h]
			\begin{center}				
				\begin{circuitikz}
					\draw (-1,0) to [short, *-] (0,0) to [european resistor, l_ = $11.25 \unit{\Omega}$] (2,0) to [L, l_ = $j45 \unit{\Omega}$] (4,0) to [short] (6,0) to [european resistor, l_ = $11.25 \unit{\Omega}$] (8,0) to [L, l_ = $j45 \unit{\Omega}$] (10, 0) to [short] (12, 0) to [european resistor, l_ = $Load$] (12,-4) to [short, -*] (-1,-4);
					\draw (5,0) to [C, l_=\text{$j 1.5 \times 10^{-3} \unit{\Omega}^{-1}$}] (5,-4);
					\draw (5,0) to [short, i_ = $\dot{I}_R$] (6.4,0);
					\draw (-1,0) to [short, i_ = $\dot{I}_S$] (0.3,0);
					\draw (5,-2.5) to [short, i_ = $\dot{I}_C$] (5,-4);
					\draw (0,0) to [open, l_= \text{$\dot{V}_S$}] (0,-4);
					\draw (10,0) to [open, l_= \text{$\dot{V}_R$}] (10,-4);
					\draw[<->] (0,-.2) -- (0,-3.8);% node[below] {$\dot{V}_S$};	
					\draw[<->] (10,-.2) -- (10,-3.8);% node[below] {$\dot{V}_R$};					
				\end{circuitikz}
			\end{center}
			\caption{Mạch tương đương hình $T$ cho đường dây trung bình trong bài tập \ref{Ex-tham-so-duong-day:bt2}} \label{Fig:mach-tuong-duong-duong-day-trung-binh-T-bt2}
			\end{figure}
			
				\item \emph{Thông số $\overline{A}, \overline{B}, \overline{C}, \overline{D}$ cho mô hình $T$ chuẩn}				
					\begin{align*}
						\overline{A} 	& = \overline{D}  = 1 + \dfrac{\overline{Y}.\overline{Z}}{2} = 1 + \dfrac{1.5 \times 10^{-3} \angle 90^0 \times 92.77 \angle 75.96^0}{2} = 0.93 \angle 1.04^0\\
						 \overline{B} & = \overline{Z} \pfm{1+\dfrac{\overline{Y}.\overline{Z}}{4}}=92.77 \angle 75.96^0  \pfm{1 + \dfrac{1.5 \times 10^{-3} \angle 90^0 \times 92.77 \angle 75.96^0}{4}} \\ 
						 & \hspace{3.15cm}= 89.64 \angle 76.46^0 \unitp{\Omega}\\
						 \overline{C} & = \overline{Y} = 1.5 \times 10^{-3} \angle 90^0 \unitp{\Omega^{-1}}
					\end{align*}				
				
				\item \emph{Xác định điện áp đầu gửi $\overline{V}_S$ và dòng điện đầu gửi $\overline{I}_S$}
					\begin{itemize}
						\item Ta có: 
							\begin{align*}
								\left[{\begin{array}{c}
								\overline{V}_S\\
								\overline{I}_S
								\end{array}
								}\right]
								= 
								\left[{\begin{array}{cc}
								\overline{A} & \overline{B}\\
								\overline{C} & \overline{D}
								\end{array}
								}\right]				
							\left[{\begin{array}{c}
							\overline{V}_R\\
							\overline{I}_R
							\end{array}
							}\right]
							= \left[{\begin{array}{cc}
								 0.93 \angle 1.04^0 & 89.64 \angle 76.46^0\\
								1.5 \times 10^{-3} \angle 90^0 &  0.93 \angle 1.04^0
								\end{array}
								}\right]				
							\left[{\begin{array}{c}
							63.51 \angle 0^0\\
							0.26 \angle -36.87^0
							\end{array}
							}\right]
						\end{align*}
						
						\item Điện áp đầu gửi:
							\begin{align*}
								\overline{V}_S  & = \overline{A}. \overline{V}_R + \overline{B}.\overline{I}_R  = 0.93 \angle 1.04^0 \times 63.51 \angle 0^0+ 89.64 \angle 76.46^0 \times 0.26 \angle -36.87^0\\
								& \hspace{3cm}= 78.64 \angle 11.68^0 \unitp{kV}
							\end{align*}
						
						\item Điện áp dây đầu gửi: $V_{LS} = \sqrt{3} \times V_S = \sqrt{3} \times 78.64 = 136.21 \unitp{kV}$.						
						
						\item Dòng điện đầu gửi:
							\begin{align*}
								\overline{I}_S & = \overline{C}. \overline{V}_R + \overline{D}.\overline{I}_R = 1.5 \times 10^{-3} \angle 90^0 \times 63.51 \angle 0^0 + 0.93 \angle 1.04^0 \times 0.26 \angle -36.87^0 \\ 
								& \hspace{3cm}= 0.20 \angle -13.28^0 \unitp{kA}
							\end{align*}
					\end{itemize}					
				
				\item \emph{Xác định góc lệch pha giữa điện áp đầu gửi và đầu nhận $\Delta \varphi_V$}
						\begin{align*}
							\Delta \varphi_V = \varphi_{V_S} - \varphi_{V_R} = 11.68^0 - 0^0 = 11.68^0
						\end{align*}
					
				\item \emph{Xác định hệ số công suất đầu gửi $\cos \varphi_S$}
					\begin{align*}
						\cos \varphi_S  = \cos \pfm{\varphi_{V_s} - \varphi_{I_{s}}}= \cos \left[{11.68^0 - \pfm{-13.28^0}} \right] = \cos 24.96^0 = 0.91	
					\end{align*}							
				
				\item \emph{Xác định công suất tác dụng, phản kháng và biểu kiến ở đầu gửi $P_S, Q_S, S_S$}
					\begin{align*}
						&\overline{S}_S  = 3 \overline{V}_S.\overline{I}_S^{\ast} = 3 \times 78.64 \angle 11.68^0 \times 0.20 \angle +13.28^0 = 42.78 + j19.91 \unitp{MVA}\\
						\Longrightarrow & P_S  = 42.78 \unit{MW}; \qquad Q_S = 19.91 \unit{MVAr}
					\end{align*}
					
				\item \emph{Xác định tổn thất công suất $\Delta P$}
					\begin{align*}
						\Delta P = P_S - P_R = 42.78 - 40 = 2.78 \unitp{MW}
					\end{align*}
									
				\item \emph{Xác định độ sụt áp $\Delta U$}
					\begin{align*}
						\Delta U \%= \dfrac{V_S - V_R}{V_R} \times 100 \% = \dfrac{78.64 - 63.51}{63.51} \times 100 \% = 23.82\%
					\end{align*}		
				
				\item \emph{Xác định hiệu suất $\eta$}
						\begin{align*}
							\eta = \dfrac{P_R}{P_S} \times 100 \% = \dfrac{40}{42.78} \times 100\% = 93.50\%
						\end{align*}	
						
				\item \emph{Vẽ giản đồ vector:} hình \ref{Fig:gian-do-vector-bt2}.
					\begin{figure}[!h]
						\begin{center}
							\begin{tikzpicture}
								\coordinate (origo) at (0,0);
								\coordinate (pivot) at (1,5);
    
    							\draw (0,0) node[left] {$0$};
    							\draw[thick,->] (origo) -- ++(4,0) node (VR) [right] {$V_R$};

								\draw[thick,->] (origo) -- ++(323.13:3) coordinate (IR) node[right] {$I_R$};
								
								\draw[thick,->] (origo) -- ++(346.72:2.3) coordinate (IS) node[right] {$I_S$};
    
							    \draw[thick,->] (origo) -- ++(380:4.95) coordinate (VS) node[right] {$V_S$};

								\pic [draw, <->, "$\varphi_R$", angle eccentricity=1.3, angle radius=1.2cm] {angle = IR--origo--VR};
    
								\pic [draw, <->, "$\varphi_S$", angle eccentricity=1.2, angle radius=2cm] {angle = IS--origo--VS};
    
							    \pic [draw, <->, "$\Delta \varphi_V$", angle eccentricity=1.2, angle radius=3cm] {angle = VR--origo--VS};
							\end{tikzpicture} \hspace{1cm}							
							 \begin{tikzpicture}
    							\coordinate (origo) at (0,0);
   								\coordinate (pivot) at (1,5);
    
    							\draw (0,0) node[left] {$0$};
							    \draw[thick,->] (origo) -- ++(4,0) node (IR) [right] {$I_R$};

								\draw[thick,->] (origo) -- ++(383.59:2.3) coordinate (IS) node[right] {$I_S$};
								
							    \draw[thick,->] (origo) -- ++(396.87:3) coordinate (VR) node[right] {$V_R$};
    
							    \draw[thick,->] (origo) -- ++(416.87:3.71) coordinate (VS) node[right] {$V_S$};

							    \pic [draw, <->, "$\varphi_R$", angle eccentricity=1.5, angle radius=.7cm] {angle = IR--origo--VR};
    
							    \pic [draw, <->, "$\varphi_S$", angle eccentricity=1.2, angle radius=1.5cm] {angle = IS--origo--VS};
    
								\pic [draw, <->, "$\Delta \varphi_V$", angle eccentricity=1.2, angle radius=2.5cm] {angle = VR--origo--VS};
						  \end{tikzpicture}
						\end{center}
						\caption{Giản đồ vector cho bài tập \ref{Ex-tham-so-duong-day:bt2}} \label{Fig:gian-do-vector-bt2}
					\end{figure}
													
			\end{enumerate}
			\begin{enumerate}[\it a.]
				\item[$\star$] \textbf{Mô hình $\mathbf{\Pi}$ chuẩn}
				\item \emph{Sơ đồ tương đường hình $\Pi$ chuẩn:} hình \ref{Fig:mach-tuong-duong-duong-day-trung-binh-Pi-bt2}.
			\begin{figure}[!h]
			\begin{center}				
				\begin{circuitikz}
					\draw(-5,0) to [short,*-] (0,0) to [european resistor, l_ = $22.5 \unit{\Omega}$] (3,0) to [L, l_ = $j90 \unit{\Omega}$] (6,0) to [short] (8,0) to [short, i_ = $ $, l_ = \text{$\dot{I}_{R}$}] (10,0) to [european resistor, l_ = $Load$] (10,-4) to [short, -*] (-5,-4);
					\draw (6,0) to [C, l_=\text{$j0.75 \times 10^{-3} \unit{\Omega}^{-1}$}] (6,-4);
					\draw (6,-2.5) to [short, i_ = $\dot{I}_C$] (6,-4);
					\draw (0,0) to [C, l_=\text{$j0.75 \times 10^{-3} \unit{\Omega}^{-1}$}] (0,-4);
					\draw (-4,0) to [open, l_= \text{$\dot{V}_S$}] (-4,-4);
					\draw (8,0) to [open, l_= \text{$\dot{V}_R$}] (8,-4);
					\draw[<->] (-4,-.2) -- (-4,-3.8);% node[below] {$\dot{V}_S$};	
					\draw[<->] (8,-.2) -- (8,-3.8);%node[below] {$\dot{V}_R$};					
				\end{circuitikz}
			\end{center}
			\caption{Mạch tương đương hình $\Pi$ cho đường dây trung bình trong bài tập \ref{Ex-tham-so-duong-day:bt2}} \label{Fig:mach-tuong-duong-duong-day-trung-binh-Pi-bt2}
			\end{figure}
			
				\item \emph{Thông số $\overline{A}, \overline{B}, \overline{C}, \overline{D}$ cho mô hình $\Pi$ chuẩn}				
					\begin{align*}
						\overline{A} 	& = \overline{D}  = 1 + \dfrac{\overline{Y}.\overline{Z}}{2} = 1 + \dfrac{1.5 \times 10^{-3} \angle 90^0 \times 92.77 \angle 75.96^0}{2} = 0.93 \angle 1.04^0\\
						 \overline{B} & = \overline{Z} =92.77 \angle 75.96^0  \unitp{\Omega} \\ 
						 \overline{C} & = \overline{Y} \pfm{1+\dfrac{\overline{Y}.\overline{Z}}{4}}= 1.5 \times 10^{-3} \angle 90^0  \pfm{1 + \dfrac{1.5 \times 10^{-3} \angle 90^0 \times 92.77 \angle 75.96^0}{4}} \\ 
						 & = 1.45 \times 10^{-3} \angle 90.5^0\unitp{\Omega^{-1}}
					\end{align*}				
				
				\item \emph{Xác định điện áp đầu gửi $\overline{V}_S$ và dòng điện đầu gửi $\overline{I}_S$}
					\begin{itemize}
						\item Ta có: 
							\begin{align*}
								\left[{\begin{array}{c}
								\overline{V}_S\\
								\overline{I}_S
								\end{array}
								}\right]
								= 
								\left[{\begin{array}{cc}
								\overline{A} & \overline{B}\\
								\overline{C} & \overline{D}
								\end{array}
								}\right]				
							\left[{\begin{array}{c}
							\overline{V}_R\\
							\overline{I}_R
							\end{array}
							}\right]
							= \left[{\begin{array}{cc}
								 0.93 \angle 1.04^0 & 92.77 \angle 75.96^0\\
								1.45 \times 10^{-3} \angle 90.5^0 &  0.93 \angle 1.04^0
								\end{array}
								}\right]				
							\left[{\begin{array}{c}
							63.51 \angle 0^0\\
							0.26 \angle -36.87^0
							\end{array}
							}\right]
						\end{align*}
						
						\item Điện áp pha đầu gửi:
							\begin{align*}
								\overline{V}_S & = \overline{A}. \overline{V}_R + \overline{B}.\overline{I}_R = 0.93 \angle 1.04^0 \times 63.51 \angle 0^0+ 92.77 \angle 75.96^0 \times 0.26 \angle -36.87^0 \\
								& \hspace{3cm}= 79.46 \angle 11.82^0 \unitp{kV}
							\end{align*}
							
						\item Điện áp dây đầu gửi: $V_{LS} = \sqrt{3} \times V_S = \sqrt{3} \times 79.46 = 137.63 \unitp{kV}$.
						\item Dòng điện đầu gửi:
							\begin{align*}						
								\overline{I}_S & = \overline{C}. \overline{V}_R + \overline{D}.\overline{I}_R= 1.45 \times 10^{-3} \angle 90.5^0 \times 63.51 \angle 0^0 + 0.93 \angle 1.04^0 \times 0.26 \angle -36.87^0 \\ 
								&\hspace{3cm} = 0.20 \angle -14.22^0 \unitp{kA}
							\end{align*}
					\end{itemize}					
				
				\item \emph{Xác định góc lệch pha giữa điện áp đầu gửi và đầu nhận $\Delta \varphi_V$}
						\begin{align*}
							\Delta \varphi_V = \varphi_{V_S} - \varphi_{V_R} = 11.82^0 - 0^0= 11.82^0
						\end{align*}
					
				\item \emph{Xác định hệ số công suất đầu gửi $\cos \varphi_S$}
					\begin{align*}
						\cos \varphi_S  = \cos \pfm{\varphi_{V_s} - \varphi_{I_{s}}}= \cos \left[{11.82^0 - \pfm{-14.22^0}} \right] = \cos 26.04^0 = 0.90
					\end{align*}					
				
				\item \emph{Xác định công suất tác dụng, phản kháng và biểu kiến ở đầu gửi $P_S, Q_S, S_S$}
					\begin{align*}
						& \overline{S}_S  = 3 \overline{V}_S.\overline{I}_S^{\ast} = 3 \times 79.46 \angle 11.82^0 \times 0.20 \angle +14.22^0 =  42.84+ j20.93 \unitp{MVA}\\
						\Longrightarrow & P_S  = 42.84 \unit{MW}; \qquad Q_S = 20.93 \unit{MVAr}
					\end{align*}						
			
				\item \emph{Xác định tổn thất công suất $\Delta P$}
					\begin{align*}
						\Delta P = P_S - P_R = 42.84 - 40 = 2.84 \unitp{MW}
					\end{align*}				
				
				\item \emph{Xác định độ sụt áp $\Delta U$}
					\begin{align*}
						\Delta U \%= \dfrac{V_S - V_R}{V_R} \times 100 \% = \dfrac{79.46 - 63.51}{63.51} \times 100 \% = 25.11\%
					\end{align*}		
				
				\item \emph{Xác định hiệu suất $\eta$}
						\begin{align*}
							\eta = \dfrac{P_R}{P_S} \times 100 \% = \dfrac{40}{42.84} \times 100\% = 93.37\%
						\end{align*}
						
				\item \emph{Vẽ giản đồ vector:} hình \ref{Fig:gian-do-vector-bt2} trang \pageref{Fig:gian-do-vector-bt2}.			
			\end{enumerate}
			
			\begin{enumerate}[\it a.]
				\item[$\star$] \textbf{Mô hình tổng quát}
				
				\item \emph{Sơ đồ tương đương cho đường dây dài:} hình \ref{Fig:mach-tuong-duong-duong-day-dai-bt2}.
			\begin{figure}[!h]
			\begin{center}				
				\begin{circuitikz}
					\draw (0,0) to [short, *-, i_ = $I_S$] (1, 0) to [european resistor, l_ = $0.15 \unit{\Omega}$] (3,0) to [L, l_ = $j0.6 \unit{\Omega}$] (5,0) to [short] (6.5,0) to [european resistor, l_ = $0.15 \unit{\Omega}$] (8.5,0) to [L, l_ = $j0.6 \unit{\Omega}$] (10.5,0) to [short] (12,0) to [european resistor, l_ = $0.15 \unit{\Omega}$] (14,0) to [L, l_ = $j0.6 \unit{\Omega}$] (16,0) to [short, i_ = $\dot{I}_S$] (18,0) to [european resistor, l_=$Load$] (18,-4) to [short, -*] (0,-4);
					\draw (5.75, 0) to [short] (5.75,-1) to [short] (7.75,-1) to [C, l_ = $j10\times 10^{-6} \unit{\Omega}$] (7.75,-3) to [short] (3.75,-3) to [european resistor, l_=$g$] (3.75,-1) to [short] (5.75,-1);
					\draw (5.75,-3) to [short] (5.75,-4);
					
					\draw (11.25, 0) to [short] (11.25,-1) to [short] (13.25,-1) to [C, l_ = $j10\times 10^{-6} \unit{\Omega}$] (13.25,-3) to [short] (9.25,-3) to [european resistor, l_=$g$] (9.25,-1) to [short] (11.25,-1);
					\draw (11.25,-3) to [short] (11.25,-4);
					
					\draw (1,0) to [open, l_= \text{$\dot{V}_S$}] (1,-4);
					\draw (16,0) to [open, l_= \text{$\dot{V}_R$}] (16,-4);
					\draw[<->] (1,-.2) -- (1,-3.8);% node[below] {$\dot{V}_S$};	
					\draw[<->] (16,-.2) -- (16,-3.8);%node[below] {$\dot{V}_R$};					
				\end{circuitikz}
			\end{center}
			\caption{Mạch thông số rải cho đường dây trung bình trong bài tập \ref{Ex-tham-so-duong-day:bt2}} \label{Fig:mach-tuong-duong-duong-day-dai-bt2}
			\end{figure}
			
				\item \emph{Thông số $\overline{A}, \overline{B}, \overline{C}, \overline{D}$ cho mô hình tổng quát}				
					\begin{align*}
						\overline{A} 	& = \overline{D}  = 1 + \dfrac{\overline{Y}.\overline{Z}}{2} + \dfrac{\overline{Y}^2.\overline{Z}^2}{24}\\
						 & = 1 + \dfrac{1.5 \times 10^{-3} \angle 90^0 \times 92.77 \angle 75.96^0}{2} + \dfrac{\pfm{1.5 \times 10^{-3} \angle 90^0}^2 . \pfm{92.77 \angle 75.96^0}^2}{24}\\ 
						 &= 0.93 \angle 1.01^0\\
						 \overline{B} & =\overline{Z}\pfm{1 + \dfrac{\overline{Y}.\overline{Z}}{6} + \dfrac{\overline{Y}^2.\overline{Z}^2}{120}}\\
						 & =92.77 \angle 75.96^0 \left[{1 + \dfrac{1.5 \times 10^{-3} \angle 90^0 \times 92.77 \angle 75.96^0}{6} + \dfrac{\pfm{1.5 \times 10^{-3} \angle 90^0}^2 . \pfm{92.77 \angle 75.96^0}^2}{120}}\right]\\
						 &  = 90.70 \angle 76.29^0 \unitp{\Omega} \\ 
						 \overline{C} & = \overline{Y} \pfm{1 + \dfrac{\overline{Y}.\overline{Z}}{6} + \dfrac{\overline{Y}^2.\overline{Z}^2}{120}}\\
						 & =1.5 \times 10^{-3} \angle 90^0 \left[{1 + \dfrac{1.5 \times 10^{-3} \angle 90^0 \times 92.77 \angle 75.96^0}{6} + \dfrac{\pfm{1.5 \times 10^{-3} \angle 90^0}^2 . \pfm{92.77 \angle 75.96^0}^2}{120}}\right]\\
						 & = 1.47 \times 10^{-3} \angle 90.33^0\unitp{\Omega^{-1}}
					\end{align*}				
				
				\item \emph{Xác định điện áp đầu gửi $\overline{V}_S$ và dòng điện đầu gửi $\overline{I}_S$}
					\begin{itemize}
						\item Ta có: 
							\begin{align*}
								\left[{\begin{array}{c}
								\overline{V}_S\\
								\overline{I}_S
								\end{array}
								}\right]
								= 
								\left[{\begin{array}{cc}
								\overline{A} & \overline{B}\\
								\overline{C} & \overline{D}
								\end{array}
								}\right]				
							\left[{\begin{array}{c}
							\overline{V}_R\\
							\overline{I}_R
							\end{array}
							}\right]
							= \left[{\begin{array}{cc}
								0.93 \angle 1.01^0 & 90.70 \angle 76.29^0\\
								1.47 \times 10^{-3} \angle 90.33^0 &  0.93 \angle 1.01^0
								\end{array}
								}\right]				
							\left[{\begin{array}{c}
							63.51 \angle 0^0\\
							0.26 \angle -36.87^0
							\end{array}
							}\right]
						\end{align*}
						
						\item Điện áp pha đầu gửi:
							\begin{align*}
								\overline{V}_S & = \overline{A}. \overline{V}_R + \overline{B}.\overline{I}_R = 0.93 \angle 1.01^0\times 63.51 \angle 0^0+ 90.70 \angle 76.29^0 \times 0.26 \angle -36.87^0 \\
								& \hspace{3cm}= 78.91 \angle 11.71^0 \unitp{kV}
							\end{align*}
							
						\item Điện áp dây đầu gửi: $V_{LS} = \sqrt{3} \times V_S = \sqrt{3} \times 78.91 = 136.68 \unitp{kV}$.
						\item Dòng điện đầu gửi:
							\begin{align*}						
								\overline{I}_S & = \overline{C}. \overline{V}_R + \overline{D}.\overline{I}_R= 1.47 \times 10^{-3} \angle 90.33^0 \times 63.51 \angle 0^0 + 0.93 \angle 1.01^0 \times 0.26 \angle -36.87^0 \\ 
								&\hspace{3cm} = 0.20 \angle -13.88^0 \unitp{kA}
							\end{align*}
					\end{itemize}					
				
				\item \emph{Xác định góc lệch pha giữa điện áp đầu gửi và đầu nhận $\Delta \varphi_V$}
						\begin{align*}
							\Delta \varphi_V = \varphi_{V_S} - \varphi_{V_R} = 11.71^0 - 0^0= 11.71^0
						\end{align*}
					
				\item \emph{Xác định hệ số công suất đầu gửi $\cos \varphi_S$}
					\begin{align*}
						\cos \varphi_S  = \cos \pfm{\varphi_{V_s} - \varphi_{I_{s}}}= \cos \left[{11.71^0 - \pfm{-13.88^0}} \right] = \cos 25.59^0 = 0.90
					\end{align*}					
				
				\item \emph{Xác định công suất tác dụng, công suất biểu kiến ở đầu gửi $P_S, Q_S, S_S$}
					\begin{align*}
						& \overline{S}_S  = 3 \overline{V}_S.\overline{I}_S^{\ast} = 3 \times 78.91 \angle 11.71^0 \times 0.20 \angle +13.88^0 =  42.70+ j20.45 \unitp{MVA}\\
						\Longrightarrow & P_S  = 42.70 \unit{MW}; \qquad Q_S = 20.45 \unit{MVAr}
					\end{align*}						
			
				\item \emph{Xác định tổn thất công suất $\Delta P$}
					\begin{align*}
						\Delta P = P_S - P_R = 42.70 - 40 = 2.70 \unitp{MW}
					\end{align*}				
				
				\item \emph{Xác định độ sụt áp $\Delta U$}
					\begin{align*}
						\Delta U \%= \dfrac{V_S - V_R}{V_R} \times 100 \% = \dfrac{78.91 - 63.51}{63.51} \times 100 \% = 24.27\%
					\end{align*}		
				
				\item \emph{Xác định hiệu suất $\eta$}
						\begin{align*}
							\eta = \dfrac{P_R}{P_S} \times 100 \% = \dfrac{40}{42.70} \times 100\% = 93.68\%
						\end{align*}
						
				\item \emph{Vẽ giản đồ vector:} hình \ref{Fig:gian-do-vector-bt2} trang \pageref{Fig:gian-do-vector-bt2}.				
			\end{enumerate}
			
		\item \label{Ex-tham-so-duong-day:bt3} Một đường dây 3 pha, $345 \unit{kV}$, $f=60Hz$, dài $500 \unit{km}$ chuyển cho đầu nhận tải $200 \unit{MW}$, $\cos \varphi_R = 0.886$ (trễ). Điện trở mỗi pha trên một $km$ là $0.08 \unit{\Omega}$,  dung dẫn mỗi pha trên một $km$ là $4 \times 10^{-6} \unit{\Omega^{-1}}$ và cảm kháng mỗi pha trên một $km$ là $0.6 \unit{\Omega}$. Tính:
			\begin{enumerate}[a.]
				\item Vẽ sơ đồ thay thế cho mô hình đường dây dài.
				\item Các thông số $\overline{A}, \overline{B}, \overline{C}, \overline{D}$ của đường dây.
				\item Điện áp và dòng điện đầu gửi.
				\item Hệ số công suất đầu gửi.
				\item Góc lệnh pha giữa điện áp đầu gửi và đầu gửi.
				\item Độ sụt áp.
				\item Công suất tác dụng, phản kháng và biểu kiến ở đầu gửi.
				\item Tổn thất công suất trên đường dây.
				\item Hiệu suất truyền tải.
				\item Vẽ giản đồ vector.
				\item[$\ast$] \emph{Lưu ý:} Làm tròn kết quả \emph{2 chữ số} sau dấu phẩy.
			\end{enumerate}
			
		\subparagraph{Bài giải bài tập \ref{Ex-tham-so-duong-day:bt3}}
			\begin{enumerate}[\it a.]
				\item[$\bullet$] Ta có:
					\begin{align*}
						\overline{z} & = r_0 + j x_0 = 0.08 +j 0.6 = 0.61 \angle 82.41^0\unitp{\Omega/km}\\
						\overline{y} & = g_0 + j b_0 \approx j b_0 = j 4 \times 10^{-6} = j 2 \times 10^{-3} = 4 \times 10^{-6} \angle 90^0 \unitp{\Omega^{-1}/km}\\
						\theta & = l\sqrt{\overline{z}.\overline{y}} = 500 \times \sqrt{0.61 \angle 82.41^0 \times 4 \times 10^{-6} \angle 90^0} = 500 \times \sqrt{2.44 \times 10^{-6} \angle 172.41^0}\\
						& \hspace{1.7cm}= 500 \times \sqrt{2.44 \times 10^{-6}} \angle \dfrac{172.41^0}{2} = 0.78 \angle 86.21^0 = 0.05 + j0.78 \unit{rad}\\
						Z_C & = \sqrt{\dfrac{\overline{z}}{\overline{y}}} =  \sqrt{\dfrac{0.61 \angle 82.41^0}{4 \times 10^{-6} \angle 90^0}} =  \sqrt{152500 \angle -7.59^0} \sqrt{152500} \angle \dfrac{ -7.59^0}{2} = 390.51 \angle -3.80^0 \unit{\Omega}	
					\end{align*}
					
				\item[$\bullet$] Tính $\cosh \theta$ và $\sinh \theta$:
				\begin{align*}
					\cosh \theta & = \cosh \pfm{0.05 + j0.78} = \cosh 0.05 \cos 0.78 + j \sinh 0.05\sin 0.78 \textrm{ (đơn vị tính là rad)}\\
						&  = 0.71 + j 0.04 = 0.71 \angle 3.22^0\\
						\sinh \theta & = \sinh \pfm{0.05 + j0.78} = \sinh 0.05 \cos 0.78 + j \cosh 0.05\sin 0.78\textrm{ (đơn vị tính là  rad)}\\
						&  = 0.04 + j 0.70 = 0.70 \angle 86.73^0
				\end{align*}
	
				\item[$\bullet$] Chọn $\overline{V}_R = \dfrac{345}{\sqrt{3}} \angle 0^0 = 199.19 \angle 0^0 \unitp{kV} $
				
				\item[$\bullet$] Suy ra: $I_R = \dfrac{P_R}{\sqrt{3}V_R \cos \varphi} = \dfrac{200}{\sqrt{3} \times 345 \times 0.886} = 0.38 \unitp{kA}$.
				
				\item[$\bullet$] Có $\cos \varphi_R = 0.886$ (trễ) $ \Longrightarrow \varphi_R  = +30.00^0$.				
				\item[$\bullet$] Có $\varphi_R = \varphi_{V_R} - \varphi_{I_R} \Longrightarrow \varphi_{I_R} = \varphi_{V_R} - \varphi_R = 0^0 - 30.00^0 = -30.00^0$.
				\item[$\bullet$] Nên: $\overline{I}_R = 0.38 \angle -30.00^0 \unitp{kA}$ .
				\item \emph{Thông số $\overline{A}, \overline{B}, \overline{C}, \overline{D}$ cho mô hình tổng quát}				
					\begin{align*}
						\overline{A} & = \overline{D} = \cosh \theta = 0.71 \angle 3.22^0 \\
						\overline{B} & = \overline{Z}_C \sinh \theta = 390.51 \angle -3.80^0 \times 0.70 \angle 86.73^0 = 273.36 \angle 82.93^0 \unitp{\Omega}\\
						\overline{C} & = \dfrac{\sinh\theta}{\overline{Z}_C} = \dfrac{0.70 \angle 86.73^0}{390.51 \angle -3.80^0} = 1.79 \times 10^{-3} \angle 90.53^0
					\end{align*}
					
				\item \emph{Xác định điện áp đầu gửi $\overline{V}_S$ và dòng điện đầu gửi $\overline{I}_S$}
					\begin{itemize}
						\item Ta có: 
							\begin{align*}
								\left[{\begin{array}{c}
								\overline{V}_S\\
								\overline{I}_S
								\end{array}
								}\right]
								= 
								\left[{\begin{array}{cc}
								\overline{A} & \overline{B}\\
								\overline{C} & \overline{D}
								\end{array}
								}\right]				
							\left[{\begin{array}{c}
							\overline{V}_R\\
							\overline{I}_R
							\end{array}
							}\right]
							= \left[{\begin{array}{cc}
								0.71 \angle 3.22^0 & 273.36 \angle 82.93^0\\
								1.79 \times 10^{-3} \angle 90.53^0 & 0.71 \angle 3.22^0
								\end{array}
								}\right]				
							\left[{\begin{array}{c}
							I\\
							0.38 \angle -30.00^0
							\end{array}
							}\right]
						\end{align*}
						
						\item Điện áp đầu gửi:
							\begin{align*}
								\overline{V}_S  & = \overline{A}. \overline{V}_R + \overline{B}.\overline{I}_R = 0.71 \angle 3.22^0 \times 199.19 \angle 0^0 + 273.36 \angle 82.93^0 \times 0.38 \angle -30.00^0 \\
								& \hspace{3cm}= 223.14 \angle 24.02^0 \unitp{kV}
								\end{align*}
								
						\item Điện áp dây đầu gửi: $V_{LS}  = \sqrt{3} V_R = \sqrt{3} \times 223.14 = 386.49 \unitp{kV}$.
						\item Dòng điện đầu gửi:
							\begin{align*}								
								\overline{I}_S & = \overline{C}. \overline{V}_R + \overline{D}.\overline{I}_R = 1.79 \times 10^{-3} \angle 90.53^0 \times 199.19 \angle 0^0 + 0.71 \angle 3.22^0 \times 0.38 \angle -30.00^0 \\
								& \hspace{3cm} = 0.33 \angle 44.69^0 \unitp{kA}
							\end{align*}
					\end{itemize}

					\item \emph{Xác định hệ số công suất đầu gửi $\cos \varphi_S$}
						\begin{align*}
							\cos \varphi_S  = \cos \pfm{\varphi_{V_s} - \varphi_{I_{s}}}= \cos \left[{24.02^0 - 44.69^0} \right] = \cos \pfm{-20.67^0} = 0.94
						\end{align*}
				
				\item \emph{Xác định góc lệch pha giữa điện áp đầu gửi và đầu nhận $\Delta \varphi_V$}
					\begin{align*}
						\Delta \varphi_V = \varphi_{V_S} - \varphi_{V_R} = 24.02^0 - 0^0 = 24.02^0
					\end{align*}
					
				\item \emph{Xác định độ sụt áp $\Delta U$}
					\begin{align*}
						\Delta U \%= \dfrac{V_S - V_R}{V_R} \times 100 \% = \dfrac{223.14 - 199.19}{199.19} \times 100 \% = 12.02\%
					\end{align*}
					
				\item \emph{Xác định công suất tác dụng, phản kháng và biểu kiến ở đầu gửi $P_S, Q_S, S_S$}
					\begin{align*}
						& \overline{S}_S  = 3 \overline{V}_S.\overline{I}_S^{\ast} = 3 \times 223.14 \angle 24.02^0 \times 0.33 \angle -44.69^0 = 206.69 - j77.98 \unitp{MVA}\\
						\Longrightarrow & P_S  = 206.69 \unit{MW}; \qquad Q_S = 77.98 \unit{MVAr}
					\end{align*}
				\item \emph{Xác định tổn thất công suất $\Delta P$}
					\begin{align*}
						\Delta P = P_S - P_R = 206.69 - 200 = 6.69 \unitp{MW}
					\end{align*}
				
				\item \emph{Xác định hiệu suất $\eta$}
						\begin{align*}
							\eta = \dfrac{P_R}{P_S} \times 100 \% = \dfrac{200}{206.69} \times 100\% = 96.76\%
						\end{align*}
					
				\item \emph{Vẽ giản đồ vector:} hình \ref{Fig:gian-do-vector-bt3}.
					\begin{figure}[!h]
						\begin{center}
							\begin{tikzpicture}
								\coordinate (origo) at (0,0);
								\coordinate (pivot) at (1,5);
    
    							\draw (0,0) node[left] {$0$};
    							\draw[thick,->] (origo) -- ++(4,0) node (VR) [right] {$V_R$};

								\draw[thick,->] (origo) -- ++(330:3) coordinate (IR) node[right] {$I_R$};
								
								\draw[thick,->] (origo) -- ++(405:2.6) coordinate (IS) node[right] {$I_S$};
    
							    \draw[thick,->] (origo) -- ++(384:4.48) coordinate (VS) node[right] {$V_S$};

								\pic [draw, <->, "$\varphi_R$", angle eccentricity=1.3, angle radius=1.2cm] {angle = IR--origo--VR};
    
								\pic [draw, <->, "$\varphi_S$", angle eccentricity=1.2, angle radius=1.5cm] {angle = VS--origo--IS};
    
							    \pic [draw, <->, "$\Delta \varphi_V$", angle eccentricity=1.2, angle radius=2cm] {angle = VR--origo--VS};
							\end{tikzpicture} \hspace{1cm}							
							 \begin{tikzpicture}
    							\coordinate (origo) at (0,0);
   								\coordinate (pivot) at (1,5);
    
    							\draw (0,0) node[left] {$0$};
							    \draw[thick,->] (origo) -- ++(4,0) node (IR) [right] {$I_R$};

								\draw[thick,->] (origo) -- ++(435:2.6) coordinate (IS) node[right] {$I_S$};
								
							    \draw[thick,->] (origo) -- ++(390:3) coordinate (VR) node[right] {$V_R$};
    
							    \draw[thick,->] (origo) -- ++(414:3.71) coordinate (VS) node[right] {$V_S$};

							    \pic [draw, <->, "$\varphi_R$", angle eccentricity=1.5, angle radius=.7cm] {angle = IR--origo--VR};
    
							    \pic [draw, <->, "$\varphi_S$", angle eccentricity=1.2, angle radius=1.3cm] {angle = VS--origo--IS};
    
								\pic [draw, <->, "$\Delta \varphi_V$", angle eccentricity=1.2, angle radius=2cm] {angle = VR--origo--VS};
						  \end{tikzpicture}
						\end{center}
						\caption{Giản đồ vector cho bài tập \ref{Ex-tham-so-duong-day:bt3}} \label{Fig:gian-do-vector-bt3}
					\end{figure}
			\end{enumerate}
	\end{enumerate}